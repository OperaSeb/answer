You can create a file in C++ using the `std::ofstream` class. Here's an example:

```cpp
#include <iostream>
#include <fstream>

int main() {
    std::ofstream file("example.txt"); // create file example.txt
    if (file.is_open()) { // check if file was opened successfully
        file << "This is an example file." << std::endl;
        file.close(); // close the file
        std::cout << "File created successfully." << std::endl;
    }
    else {
        std::cout << "Failed to create file." << std::endl;
    }
    return 0;
}
```

In this example, we create an `std::ofstream` object named `file` and pass the name of the file we want to create ("example.txt") to the constructor. We then check if the file was opened successfully using the `is_open()` method, and if so, we write some text to the file using the output stream operator `<<`. Finally, we close the file using the `close()` method.

Note that if the file already exists, the `std::ofstream` constructor will truncate it to zero length by default. If you want to append to an existing file, you can open it using `std::ofstream::app` flag like this:

```cpp
std::ofstream file("example.txt", std::ofstream::app);
```

This will open the file in "append mode", which means that new data will be added to the end of the file instead of overwriting its contents.
