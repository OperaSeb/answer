Sure, here are some examples of parsing different file formats (CML, JSON, and CSV) in Rust:

Parsing CML:

```rust
use cml::parser::parse;
use std::fs;

fn main() {
    let content = fs::read_to_string("file.cml").unwrap();
    let result = parse(&content);
    println!("{:#?}", result);
}
```

Here, we're using the `cml` crate to parse a CML file. We first read the contents of the file into a string, then pass that string to the `parse` function, which returns a `Result` containing the parsed CML data. We then print the result using the `println!` macro.

Parsing JSON:

```rust
use serde::{Deserialize, Serialize};
use std::fs;

#[derive(Debug, Deserialize, Serialize)]
struct Person {
    name: String,
    age: u32,
}

fn main() {
    let content = fs::read_to_string("file.json").unwrap();
    let result: Vec<Person> = serde_json::from_str(&content).unwrap();
    println!("{:#?}", result);
}
```

Here, we're using the `serde_json` crate to parse a JSON file. We first read the contents of the file into a string, then use `serde_json::from_str` to parse the JSON data into a `Vec<Person>` struct. We then print the result using the `println!` macro.

Parsing CSV:

```rust
use csv::ReaderBuilder;
use std::fs;

fn main() {
    let file = fs::File::open("file.csv").unwrap();
    let mut csv_reader = ReaderBuilder::new()
        .has_headers(true)
        .from_reader(file);
    for result in csv_reader.records() {
        let record = result.unwrap();
        println!("{:?}", record);
    }
}
```

Here, we're using the `csv` crate to parse a CSV file. We first open the file using `fs::File::open`, then pass the file to a `Reader` instance created using `ReaderBuilder`. We set `has_headers` to `true` to indicate that the CSV file has a header row, and then iterate over the records using the `records` method. For each record, we print it using the `println!` macro.
