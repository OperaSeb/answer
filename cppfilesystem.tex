Sure, here is an example of using the `std::filesystem::path` class in C++:

```
#include <iostream>
#include <filesystem>

int main() {
    std::filesystem::path path1("/home/user/Documents");
    std::filesystem::path path2("file.txt");

    // Combine path1 and path2 using the '/' operator
    std::filesystem::path combinedPath = path1 / path2;

    // Output the combined path
    std::cout << "Combined path: " << combinedPath << std::endl;

    // Output the filename component of the combined path
    std::cout << "Filename: " << combinedPath.filename() << std::endl;

    // Output the parent path of the combined path
    std::cout << "Parent path: " << combinedPath.parent_path() << std::endl;

    // Check if the combined path exists
    if (std::filesystem::exists(combinedPath)) {
        std::cout << "The combined path exists." << std::endl;
    }
    else {
        std::cout << "The combined path does not exist." << std::endl;
    }

    return 0;
}
```

In this example, we first create two `std::filesystem::path` objects, `path1` and `path2`, representing a directory and a file respectively. We then combine these paths using the `operator/` overload to create a new path `combinedPath`. 

We then output the filename component and the parent path of the `combinedPath` using the `filename()` and `parent_path()` member functions respectively.

Finally, we check if the `combinedPath` exists using the `std::filesystem::exists()` function.

Note that the `std::filesystem` library was introduced in C++17 and may not be available on older compilers. Also, the behavior of some member functions may vary depending on the operating system and file system being used.
