	This is an if statement in a bash script that sets the value of the PS1 variable based on the value of the color_prompt variable.

Here's how the statement works:

if [ "$color_prompt" = yes ]; then: This is the start of the if statement. It checks whether the color_prompt variable is set to the string "yes". If it is, the following block of code will be executed. Note that the [ and ] characters are shell built-in commands that test conditions, and they must be surrounded by spaces.
PS1='${debian_chroot:+($debian_chroot)}\[\033[01;32m\]\u@\h\[\033[00m\]:\[\033[01;34m\]\w\[\033[00m\]\$ ': If color_prompt is "yes", this line sets the value of the PS1 variable to a string that contains several escape sequences to control the appearance of the command prompt. The escape sequences are enclosed in \[\] to indicate that they should not be counted when calculating the length of the prompt. The prompt includes the username (\u), hostname (\h), current working directory (\w), and a $ character to indicate the end of the prompt.
else: This is the start of the else block, which will be executed if color_prompt is not "yes".
PS1='${debian_chroot:+($debian_chroot)}\u@\h:\w\$ ': If color_prompt is not "yes", this line sets the value of PS1 to a simpler string that includes the username, hostname, current working directory, and $ character, but does not use any escape sequences to change the appearance of the prompt.
Overall, this statement is used to set the value of the PS1 variable based on the value of color_prompt. If color_prompt is "yes", the prompt will include escape sequences to display the username, hostname, and working directory in color. Otherwise, the prompt will display these items in plain text.
