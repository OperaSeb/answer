Here is a simple example of using a `std::unordered_map` in C++ to store and retrieve key-value pairs:

```cpp
#include <iostream>
#include <unordered_map>

int main() {
    std::unordered_map<std::string, int> myMap;
    myMap["foo"] = 42;
    myMap["bar"] = 123;

    std::cout << "foo: " << myMap["foo"] << std::endl;
    std::cout << "bar: " << myMap["bar"] << std::endl;

    return 0;
}
```

In this example, we create an unordered map `myMap` with `std::string` keys and `int` values. We then insert two key-value pairs into the map using the subscript operator `[]`. Finally, we retrieve and print the values of the two keys using the subscript operator `[]` again.

The output of this program should be:
```
foo: 42
bar: 123
```

Note that `std::unordered_map` is an implementation of a hash table, which provides fast access to elements based on their keys. The `[]` operator is used to both insert and retrieve elements in the map. If the key is not already present in the map, the `[]` operator inserts a new key-value pair into the map. If the key is already present, the `[]` operator returns a reference to the corresponding value, allowing it to be updated.
