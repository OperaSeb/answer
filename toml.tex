In programming, a TOML file (short for "Tom's Obvious, Minimal Language") is a file format that is used for configuration files. It is designed to be easy to read and write by humans, while also being easy to parse by computers.

TOML is a relatively new file format that was first introduced in 2013 as an alternative to other configuration file formats such as JSON, XML, and YAML. It is often used in Rust projects and has gained popularity in other programming communities as well.

TOML files are organized into sections, which are delimited by square brackets. Each section can contain key-value pairs, where the keys are strings and the values can be strings, numbers, boolean values, arrays, or nested tables.

Here's an example of a simple TOML file:

makefile
Copy code
[server]
host = "localhost"
port = 8080

[database]
name = "my_database"
user = "my_user"
password = "my_password"
In this example, there are two sections: [server] and [database]. Each section contains a set of key-value pairs that define the configuration options for the server and the database, respectively.

TOML is often praised for its simplicity and readability, as well as its ability to handle complex data structures in a concise and intuitive way.
