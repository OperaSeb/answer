I believe you are referring to the Tor (The Onion Router) network, which is a free and open-source software used to protect users' online privacy and anonymity by providing a way to access the internet through a series of encrypted tunnels that bounce your internet traffic between different servers, making it difficult to track the origin and destination of the communication.

When you use the Tor browser, it sends your internet traffic through a network of volunteer-run servers, known as nodes, that are distributed worldwide. These nodes act as relays, encrypting your data as it passes through them, making it challenging for anyone to trace your internet activity back to your location.

Each relay only knows about the node it receives data from and the node it sends data to, ensuring that no single relay knows the entire path of the communication. Moreover, Tor uses a technique called onion routing, which encrypts the data multiple times, making it difficult for anyone to decipher the content or trace its origin.

The Tor network also provides a way to access hidden services, which are websites that are only accessible through the Tor network, and they are designed to protect the privacy of both the website operator and the users who access them.

Overall, the Tor network provides a way to protect your online privacy and anonymity, but it is not foolproof, and there are still potential vulnerabilities that could be exploited. It is important to keep your software updated, use a reliable VPN service, and follow security best practices when using the Tor network or any other tools designed for online anonymity.
