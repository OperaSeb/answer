You can read a file in Rust using the `File` type and the `BufReader` type from the standard library. Here's an example:

```rust
use std::fs::File;
use std::io::{BufRead, BufReader};

fn main() -> std::io::Result<()> {
    let file = File::open("example.txt")?;
    let reader = BufReader::new(file);

    for line in reader.lines() {
        println!("{}", line?);
    }

    Ok(())
}
```

In this example, we open the file named "example.txt" using the `File::open` method. The method returns a `Result<File>` that can be used to handle any errors that occur during the file opening. We then create a `BufReader` instance from the file, which allows us to read the file line-by-line using the `lines` method. 

We use a `for` loop to iterate over the lines returned by `lines()`, and we print each line using `println!()`. The `?` operator is used to propagate any errors that occur during the line reading process.

Finally, we return an `Ok(())` value to indicate that the operation completed successfully. If an error occurred during the file opening or reading, we would return a `Err` value containing the error information.
