Sure, here is a simple example of using a `HashMap` in Rust:

```rust
use std::collections::HashMap;

fn main() {
    let mut map = HashMap::new();

    // insert some key-value pairs into the map
    map.insert("one", 1);
    map.insert("two", 2);
    map.insert("three", 3);

    // access values by key
    println!("The value of 'one' is: {:?}", map.get("one"));
    println!("The value of 'two' is: {:?}", map.get("two"));
    println!("The value of 'three' is: {:?}", map.get("three"));

    // iterate over the key-value pairs
    for (key, value) in &map {
        println!("{}: {}", key, value);
    }
}
```

In this example, we create a new `HashMap` using the `HashMap::new()` method. We then insert some key-value pairs using the `insert` method, and we can access the values by their keys using the `get` method. Finally, we iterate over the key-value pairs using a `for` loop and the `iter` method.
