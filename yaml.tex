In programming, a YAML file (short for "YAML Ain't Markup Language") is a file format that is used to store data in a structured and human-readable way. It is often used for configuration files, as it allows developers to define settings and parameters in a more readable format than some other formats, such as XML or JSON.

A YAML file consists of key-value pairs, which are organized using whitespace and indentation rather than traditional markup syntax. The values can be strings, numbers, booleans, or even more complex data structures like lists and dictionaries.

Here's an example of a simple YAML file:

yaml
Copy code
# This is a comment in YAML
name: John Smith
age: 35
isMarried: true
In this example, the YAML file defines three key-value pairs: name, age, and isMarried. The values associated with these keys are a string (John Smith), a number (35), and a boolean (true), respectively.

YAML files are often used in web development frameworks, such as Ruby on Rails and Django, to define database schema, routing, and other application configuration options.
