You can create a file in Rust using the `File` type from the standard library. Here's an example:

```rust
use std::fs::File;
use std::io::prelude::*;

fn main() -> std::io::Result<()> {
    let mut file = File::create("example.txt")?;
    file.write_all(b"Hello, world!")?;
    Ok(())
}
```

In this example, we create a new file named "example.txt" using the `File::create` method. The method returns a `Result<File>` that can be used to handle any errors that occur during the file creation. We then use the `write_all` method on the file to write the byte string "Hello, world!" to the file. Finally, we return an `Ok(())` value to indicate that the operation completed successfully. If an error occurred during the file creation or writing, we would return a `Err` value containing the error information.

Note that in order to write to the file, we need to import the `std::io::prelude::*` module and call the `write_all` method on the file instance. This method writes the entire byte string to the file, and returns a `Result<()>` indicating whether the operation succeeded or failed.
